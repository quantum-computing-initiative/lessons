% \documentclass[notes]{beamer}       % print frame + notes
% \documentclass[notes=only]{beamer}   % only notes
% \documentclass{beamer}              % only frames
% \usetheme[nofirafonts]{focus}

\documentclass[12pt]{article}
\usepackage{beamerarticle}

\title[QCI Lesson 6]{QCI Lesson 6}
\subtitle{Lab: Generating Random Numbers}
\institute{Quantum Computing Initiative}
\date{\today}

\begin{document}

\begin{frame}
  \titlepage
\end{frame}

\begin{frame}
  \frametitle{Simulating a Coin Flip}

\begin{itemize}
  \item Pick a random whole number between 0 and 1, inclusive
  \item Show coin flip in Python using \texttt{random.randint(0, 1)}
\end{itemize}

\end{frame}

\begin{frame}
  \frametitle{Pseudorandom vs. Random}

\begin{itemize}
  \item Programming languages have a \texttt{random} function that generates an \textit{apparently}
    random number—but they are not truly random. They follow a deterministic
    algorithm.
  \item These numbers are referred to as \textit{pseudorandom}, and they are generated by using
    an initial number called a \textit{seed}, which has some irreversible mathematical
    operations applied to it.
  \item (If you've played Minecraft, this is essentially what the world seed does.
    If two players input the same seed, they get the same randomly generated world)
  \item We're not going to go into how these computers
    actually generate pseudorandom numbers, but it is important to know that
    although they seem random, they are actually quite predictable if you run
    the generator enough times.
  \item In some cases, such as in cryptography, we need \textit{truly} random numbers. Quantum
    computers can do that, because quantum mechanics are fundamentally probabilistic.
\end{itemize}

\end{frame}

\begin{frame}
    \frametitle{Quantum Coin Flip}
    Simply apply a Hadamard gate to a single qubit and then run on the quantum
    computer (or simulator), passing \texttt{shots = 1} into the
    \texttt{execute()} method from Qiskit.

\end{frame}

\note[itemize]{ Sample speaker notes }

\begin{frame}
    \frametitle{Quantum Random Number Generator}

  \begin{itemize}
    \item Ask the user to input an upper bound for a random number
    \item Calculate the number of (qu)bits needed to represent that number
    \item Generate random values (coin flips!) for each qubit
    \item Convert the set of randomly generated qubits values to base 10
    \item Reject and re-generate if the result is over the upper bound provided by the user
  \end{itemize}

\end{frame}

\end{document}
